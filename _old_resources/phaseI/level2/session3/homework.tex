\documentclass{../../homework}

\usepackage{fancyvrb}
\usepackage{url}
\mytitle{Week 3, Part 2}

\begin{document}

\noindent
For some of these problems, you will submit your solutions on MarkUs
\url{https://markus.cdf.toronto.edu/c4m-2016} as you did for the homework
for workshop 1. For others you will submit on PCRS, \url{https://teach.cdf.toronto.edu/c4m} as you did for part 1 of the homework for workshop 2.

% Parallel lists
\item A patient's blood pressure is measured at every check-up, and is recorded as \verb;"low";, \verb;"high";, or \verb;"normal";. The date of each check-up is also recorded. We are interested in knowing the date on which a sequence of 3 or more \verb;"high"; measurements has started,

Write a function with the signature \verb;onset_high_blood_pressure(bp, dates); which returns the date of the first time that a sequence of three or more \verb;"high"; measurements was observed.  If there is no such sequence, return the string \verb;"No high blood pressure period observed";.

 For example, if
\begin{verbatim}
bp =    ["low",  "low", 	"high", "normal", "high",  "high", "high", "high", "normal"]
dates = ["12/3", "20/3", "29/3", "5/4",    "15/4",  "30/4", "10/5", "17/5", "29/5"]
\end{verbatim}

\verb;onset_high_blood_pressure(bp, dates); should return \verb;30/4;.

Write your function in Pyzo. Test it with the example above. Then write some other test cases for your function. One rule that is sometimes used for testing, is that your full set of tests must at least execute every line of code in your function. That
way no line goes completely untested. Make sure that your tests do this at a minimum. Put your function in a file
\verb;bp.py; and submit it on MarkUs.

\item
A patient's core body temperature is recorded at regular intervals, and stored as a list of \verb;float;s.  Write a function with the signature \verb;consistent_growth_exact_n(temps, n); that returns \verb;True; iff there was a period of \textit{exactly} n hours during which the temperature increased every hour. (I.e., if the temperature increased n+1 periods, it doesn't count.) 

Write your code on Pyzo and include the function in a file called \verb;growth.py;. In addition to the function definition,
the file should have at least 3 calls to the function on different test input values. Submit your file on MarkUs.

\begin{enumerate}

\end{document}